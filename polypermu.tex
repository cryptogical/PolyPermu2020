  
\documentclass[12pt]{article}
\usepackage[francais]{babel}
\usepackage{natbib}
\usepackage{url}
\usepackage[utf8x]{inputenc}
\usepackage{graphicx}
\graphicspath{{images/}}
\usepackage{parskip}
\usepackage{fancyhdr}
\usepackage{vmargin}
\usepackage{xcolor}
\usepackage{bbm}
\usepackage{amsmath,amssymb}
\usepackage{amsthm}


\title{Polynômes de Permutations}
\author{PIARD A. - JACQUET R. - CARVAILLO T.}
\date{\today}

\makeatletter
\let\thetitle\@title
\let\theauthor\@author
\let\thedate\@date
\makeatother

\pagestyle{fancy}
\fancyhf{}
\rhead{\theauthor}
\lhead{\thetitle}
\cfoot{\thepage}
\def\dotfill#1{\cleaders\hbox to #1{.}\hfill}
\newcommand\dotline[2][.5em]{\leavevmode\hbox to #2{\dotfill{#1}\hfil}}

\newcommand{\M}{\mathbbm{M}}
\newcommand{\N}{\mathbbm{N}}
\newcommand{\Z}{\mathbbm{Z}}
\newcommand{\Q}{\mathbbm{Q}}
\newcommand{\R}{\mathbbm{R}}
\newcommand{\C}{\mathbbm{C}}
\newcommand{\G}{\mathbbm{G}}
\newcommand{\K}{\mathbbm{K}}
\newcommand{\F}{\mathbbm{F}}

\newtheorem{theorem}{Théorème}
\newtheorem{corollaire}{Corollaire}
\newtheorem{lemma}{Lemme}
\newtheorem{prop}{Proposition}
\theoremstyle{definition}
\newtheorem{definition}{Définition}
\newtheorem{example}{Exemple}
\newtheorem*{examples}{Exemples}
\newtheorem{exo}{Exercice}	

\begin{document}

%%%%%%%%%%%%%%%%%%%%%%%%%%%%%%%%%%%%%%%%%%%%%%%%%%%%%%%%%%%%%%%%%%%%%%%%%%%%%%%%%%%%%%%%%

\begin{titlepage}
	\centering
    \vspace*{0.5 cm}
    \textsc{\LARGE Projet de Recherche . 2020-2021}\\[1.0 cm]
    \dotline[15pt]{15cm}\\
	\includegraphics[scale = 2.2]{logo.png}
	\dotline[15pt]{15cm}\\
	\vspace{1.5cm}
	\textsc{\Large Faculté des Sciences et Techniques}\\
	\textsc{\large Master 1 - Maths. CRYPTIS}\\[1.0 cm]
	\rule{\linewidth}{0.2 mm} \\[0.4 cm]
	{ \huge \bfseries \color{blue} \thetitle}\\
	\rule{\linewidth}{0.2 mm} \\[1.5 cm]
	
	\begin{minipage}{0.4\textwidth}
		\begin{flushleft} \large
			\emph{A l'attention de :}\\
			M. NECER\\
			\phantom{a}\\
			\phantom{a}\\
		\end{flushleft}
	\end{minipage}
	\begin{minipage}{0.5\textwidth}
    	\begin{flushright} \large
		\emph{Rédigé par :}\\
		PIARD A.\\
		JACQUET R.\\
		CARVAILLO T.\\
		\end{flushright}
	\end{minipage}\\[2 cm]
\end{titlepage}
%%%%%%%%%%%%%%%%%%%%%%%%%%%%%%%%%%%%%%%%%%%%%%%%%%%%%%%%%%%%%%%%%%%%%%%%%%%%%%%%%%%%%%%%%

\tableofcontents
\pagebreak

%%%%%%%%%%%%%%%%%%%%%%%%%%%%%%%%%%%%%%%%%%%%%%%%%%%%%%%%%%%%%%%%%%%%%%%%%%%%%%%%%%%%%%%%%
\section*{Introduction}
\vfill \eject
\section{Construction des Corps Finis}
\subsection{Existence et unicité}
\vspace{12pt}
Soit $\K$ un corps quelconque et soit $\varphi$ le morphisme suivant :
\begin{center}
$
\begin{array}{l|rcl}
\varphi : & \Z & \longrightarrow & \K \\
    & n & \longmapsto & n\cdot 1_{\K}
\end{array}
$
\end{center}
\vspace{12pt}
\begin{definition}
Soit $\K$ un corps quelconque. Toute partie $\mathcal{P}$ de $\K$ vérifiant :
\begin{itemize}
\item $\mathcal{P}$ est non vide et est une partie stable pour $+$ et $\times$ de $\K$ et $\mathcal{P}$ muni des lois induites par celles de $\K$ est lui-même un corps.
\item $\mathcal{P}$ est un sous anneau de $\K$, $1 \in \mathcal{P}$ et $(p \in \mathcal{P}^{*} = \mathcal{P} - \{0 \} \Rightarrow p^{-1} \in \mathcal{P}^{*})$.
\item $\mathcal{P}$ est un sous groupe de $(\K, +)$ et $\mathcal{P}^{*}$ muni de la loi $\times$ est un sous groupe multiplicatif $(\K^{*}, \times)$.
\end{itemize}
est appelée sous-corps de $\K$.
\end{definition}
\vspace{12pt}
\begin{definition}
Soit $\K$ un corps quelconque.
\begin{itemize}
\item $\K$ est dit premier s'il ne contient aucun sous-corps strict.
\item Si $\K$ est un corps, le sous-corps de $\K$ engendré par $1_{K}$ est un corps premier, c'est le sous-corps premier de $\K$.
\end{itemize}
\end{definition}
\vspace{12pt}
Le noyau de ce morphisme est un idéal de $\Z$ et donc de la forme $k\Z$ pour $k \in \Z$. Par le premier théorème d'isomorphisme on a Im$(\varphi) \cong \Z / n \Z$. Par intégrité de $\Z / n \Z$, $n=0$ ou $n$ est un nombre premier. Si $n=0$ alors $\varphi$ est injective et donc le sous-corps premier de $\K$ est isomorphe à $\Q$. Si $n \neq 0$ alors le sous-corps premier est isomorphe à $\Z / n \Z$ et $n$ s'appelle la caractéristique de $\K$. 
%On désignera dorénavant par $\K$ un corps fini de caractéristique $p$ avec $p$ un nombre premier.
\\

\begin{definition}
Soient $L$ et $\K$ deux corps. Si $L/K$ est une extension de corps alors $L$ est un espace vectoriel sur $K$, où l'addition vectorielle est l'addition dans $L$ et la multiplication par un scalaire $K \times L$ est la restriction à $K \times L$ de la multiplication dans $L$. La dimension du $K$-espace vectoriel $L$ est appelée le degré de l'extension et est notée $[L:K]$.
\end{definition}
\vspace{12pt}
\begin{definition}
Soit $P$ un polynôme sur un corps $K$. On appelle corps de décomposition de $P$ sur $K$ une extension $L$ de $K$ telle que :
\begin{itemize}
\item dans $L[X]$, $P$ est produit de facteurs de degré $1$,
\item les racines de $P$ engendrent $L$.
\end{itemize}
\end{definition}
\vspace{12pt}
\begin{prop}
Soit $P$ un polynôme sur un corps $K$. Alors $P$ admet un corps de décomposition, unique à $K$-isomorphisme près.
\end{prop}
\vspace{30pt}
\begin{prop}\hspace{12pt}
\begin{itemize}
\item Le cardinal de $\K$ est une puissance de $p$.
\item Réciproquement, pour tout $n \in \N^{*}$, il existe un corps $\K$ de cardinal $p^n$. En outre $\K$ est unique à isomorphisme près.
%On appelle corps de décomposition de P, la plus petite extension de K contenant toutes les racines de P (item 1)
\end{itemize}
\end{prop}
%\vspace{12pt}
\begin{proof}\hspace{12pt}
\begin{itemize}
\item Puisque le sous-corps premier de $\K$ est isomorphe à $\Z / p \Z$ alors $\K$ est naturellement muni d'une structure de $\Z / p \Z$-espace vectoriel. On note $n = [ \K : \Z / p \Z ]$ alors $\# \K = \# (\Z / p \Z)^n = p^n$.
\item Soit $n \in \N^{*}$. Si $\K$ est un corps fini de cardinal $p^n$ alors $\K$ est le corps de décomposition de $X^{p^n} - X$ sur $\Z / p \Z$ : en effet, puisque pour tout $x \in \K$, $x$ est racine de $X^{p^n} - X$ donc $X^{p^n} - X$ possède ses $p^n$ racines dans $\K$.\\
Réciproquement, soit $K$ le corps de décomposition de $X^{p^n}$ sur $\Z / p \Z$. Soit $\mathcal{K}$ l'ensemble des éléments de $K$ qui sont racines de $X^{p^n} - X$. On vérifie que $\mathcal{K}$ est un sous-corps de $K$. Puisque $1_K \in \mathcal{K}$, et si $x,y \in \mathcal{K}$ alors $x^{p^n}= x$ et $y^{p^n} = y$, donc $(x+y)^{p^n} x + y$ et $(xy^{-1})^{p^n} = xy^{-1}$, si bien que $x + y, xy^{-1} \in \mathcal{K}$. Par ailleurs la dérivée formelle, $(X^{p^n} - X)' = -1$ est premier avec $X^{p^n} - X$ donc les racines de $X^{p^n} - X$ sont simples. On en déduit alors que $\# \mathcal{K} = p^n$. Finalement $K = \mathcal{K}$ est un corps à $p^n$ éléments et il est unique à isomorphisme près en vertu de l'unicité du corps de décomposition de $X^{p^n} - X$ sur $\Z / p \Z$.
\end{itemize}
\end{proof}
On notera dorénavant $\F_q$ \textbf{le} corps fini à $q = p^n$ éléments.
\subsection{Construction}
Soit $P \in \F_p [X]$ un polynôme irréductible sur $\F_p$. On note $n = $ deg$(P)$. Puisque $P$ est irréductible, l'idéal $(P)$ est donc maximal%est-ce qu'on le montre ?
. Le quotient $\F_p [X] / (P)$ est le corps de rupture de $P$ sur $\F_p$ de cardinal $p^n$. Afin de montrer que l'on peut toujours construire les corps finis nous allons montrer le résultat suivant :

%NOTFINISHED
\section{Polynômes de permutations}
Rappelons d'abord ce qu'est un polynôme dans le cas général.\\ %Est-ce qu'on n'en parlerait pas dans la première partie ? Vu qu'on évoque les polynômes irréductibles dès la partie 1
\begin{definition} %Si plus tard on parle de polynôme à plusieurs indéterminées, on fera une définition à ce moment là
Soit $K$ un ensemble non vide. On appelle polynôme en l'indéterminée $X$, toute application 
\begin{eqnarray}
P : & K & \longrightarrow K  \nonumber \\ 
& X & \longmapsto  \sum_{i=0}^n a_iX^i,  a_i \in K. \nonumber
\end{eqnarray}
\end{definition}

\begin{definition}
Soit $K$ un ensemble fini de cardinal $n\in \mathbb{N}^*$. Une permutation de $K$ est une bijection de $K$ dans $K$.
\end{definition}
\vspace{12pt}
\begin{definition}
Soit $P$ un polynôme de $\F_q [X]$. $P$ est appelé \textbf{polynôme de permutation} de $\F_q$ si et seulement si la fonction associée
\begin{eqnarray}
P : &\F _q& \longrightarrow \F _q \nonumber \\
& x & \longmapsto P(x) \nonumber
\end{eqnarray}
est une permutation, c'est à dire est bijective.
\end{definition}
\vspace{24pt}
\begin{examples}
	\begin{itemize}
On se place dans $\F_5$.\\
\item[1.] Le polynôme $X^3$ est un polynôme de permutation. En effet, l'application 
		\begin{eqnarray}
P : &\F _5& \longrightarrow \F _5 \nonumber \\
& X & \longmapsto X^3 \nonumber
		\end{eqnarray}
est clairement bijective.\\
\item[2.] Le polynôme $X^2$ n'est pas un polynôme de permutation. Considérons l'application 
		\begin{eqnarray}
P : &\F _5& \longrightarrow \F _5 \nonumber \\
& X & \longmapsto X^2. \nonumber
		\end{eqnarray}
%Cette application n'est pas injective. En effet, soit $(X,Y) \in \F_5$.$P(X)=P(Y)$ si et seulement si $X^2=Y^2$. En prenant $X=3$ et $Y=-2
Il faut montrer que cette application n'est pas bijective.
	\end{itemize}
\end{examples}
\end{document}
